\documentclass[twocolumn]{article}
\usepackage{ctex}
\usepackage{ulem}
\usepackage{hyperref}

\begin{document}
\title{Mx* Language Reference Manual} 
\maketitle

\section{用词说明}

未定义(Undefined Behavior) --- \sout{指中央还没有表态}
指规范并没有定义该情况发生时语言的表现。
初衷是为了给同学们提供一些自己发挥的空间,
在测试数据里,这些没有定义的情况是不会发生的。
例:
\begin{itemize}
\item 术语:编译器接受源代码长度如果超过1M,结果是未定义的
\item 解释:我们测试用的源代码长度不会超过1M,
可以假设测试数据不会出现这种情况。
如果学生乐意,也可以让自己的编译器支持比1M更大的文件,
或者也可以在源代码超过这个长度之后报错,拒绝编译。
\end{itemize}
总之,“未定义”=“这种情况在测试数据里不会发生”。
所以在提交新测试数据的时候,
也要注意不能出现有未定义情况的程序。

\section{程序结构}
Mx*语言由以下要素构成
\begin{itemize}
\item 函数定义(function definition)
\item 类定义(class definition)
\item main函数是Mx*程序的顶层结构(toplevel)的一部分。
程序先按照顺序初始化所有全局变量,然后从main函数开始执行。main函数没有参数,返回值为整数。
一个程序不可以没有main函数。
\item 全局变量声明
\item 接口定义(interface definition)是未定义的
\end{itemize}
例1:
\begin{verbatim}
int main() {
  println("too young too simple.");
  return 0;
}
\end{verbatim}
例2:
\begin{verbatim}
// 函数调用在顶层结构后
int main() {
  int i;
  for (i = 0; i < 3; i++) 
    angry();
  return 0;
}
void angry() {
  print("I'm angry!\n");
}
\end{verbatim}
例3:
\begin{verbatim}
int Wallace = 1 << 10;
class sometimes {
  int naive;
  void make_money() {
     this.naive++;
  }
}
int main() {
  sometimes keep = new sometimes;
  keep.naive = 0;
  while (getInt() < Wallace) { 
    keep.make_money();
  }
  return 0;
}
\end{verbatim}
\section{文法规则}

\subsection{源文件编码}
ACSII编码,区分大小写。中文字符是未定义的。

\subsection{关键字 Reserved Keywords}
\begin{verbatim}
bool int string null void
true false
if for while
break continue return
new class this
\end{verbatim}
\subsection{空白字符的处理}

空格、制表符、回车符和换行符在源文件中除了区分词素(Token)外没有其他含义。

\subsection{注释}
从// 开始到本行结束的内容都会被作为注释。
类似于/**/的注释是未定义的。

\subsection{标识符}

标识符的第一个字符必须是英文字母,
第二个字符开始可以是英文字母、数字或者下划线。
标识符区分大小写。
长度超过64个字符的标识符是未定义的。

\subsection{常量}

\subsubsection{逻辑常量}
true为真,false为假

\subsubsection{整数常量}
整数常量以十进制表示。
整数常量不设负数,负数可以由正数取负号得到。
编译器至少应该能处理大小范围在$[-2^{31},2^{31})$内的整数。

首位为0的整数常量是未定义的,大小超过上述范围的整数是未定义的。

\subsubsection{字符串常量}
字符串常量是由双引号括起来的字符串。
字符串长度最小为0,长度超过255的字符串是未定义的。
字符串中的所有字符必须是可示字符(printable character),空格或者转义字符中的一种。
转义字符有三个:$\backslash$n表示换行符,$\backslash\backslash$表示反斜杠,$\backslash$"表示双引号。
其余出现在C++语言里的转义字符是未定义的。

\subsubsection{空值常量}

null用来表示引用类型没有指向任何值。

\subsubsection{数组常量}
未定义

\section{运算符}

\subsection{算术运算符}
\begin{verbatim}
+ - * / %
\end{verbatim}
\subsection{关系运算符}
\begin{verbatim}
< > == != >= <=
\end{verbatim}
\subsection{逻辑运算符}
\begin{verbatim}
&& || !
\end{verbatim}

\subsection{位运算符}
\begin{verbatim}
<< >> ~ | ^ &
\end{verbatim}
定义右移为算术右移。例如
\begin{verbatim}
11100011 >> 3 == 11111100
\end{verbatim}
\subsection{赋值运算符}
\begin{verbatim}
=
\end{verbatim}
赋值运算符的结果是未定义的。之所以这么做,是方便大家能用两种常见的方法处理
赋值运算符:或者可以像C语言一样支持连续赋值,比如a=b=c;
也可以定义赋值运算符的结果类型是void,这样可以避免写出if (a=b)这样的代码。
其他类似于+=的augmented assignment的运算是未定义的。

\subsection{自增运算符和自减运算符}
\begin{verbatim}
++ --
\end{verbatim}
\subsection{分量运算符}
\begin{verbatim}
.
\end{verbatim}
\subsection{下标运算符}
\begin{verbatim}
[]
\end{verbatim}
\subsection{括号}
\begin{verbatim}
()
\end{verbatim}
圆括号可以用于calling functions和subexpression grouping。
\subsection{优先级}
和C语言一致。
运算符的优先级从高到低大致是:单目运算符、算术运算符、关系运算符、逻辑运算符、条件运算符、赋值运算符。

\section{数据类型}

\subsection{基础类型}

\begin{description}
\item[bool类型] 略 
\item[int类型] 略
\item[void类型] void类型是用来表示函数没有返回值的特殊类型。只能在定义函数的返回值类型时使用。
如果想说明一个函数没有参数,不必写void,直接让参数列表为空即可。
\item[string类型] 字符串类型属于引用类型。字符串本身不能改变(immutable)。
\end{description}

\subsection{复合类型}

\subsubsection{数组}
数组是可以动态创建的引用类型,长度无需在声明时确定。
数组长度超过$2^{31}-1$是未定义的。
\begin{verbatim}
string[] vec;
vec = new string[10];
\end{verbatim}
注意java声明数组时,既可以写int[] a,也可以写int a[],
但我们不支持后者。
\subsubsection{数组的内建方法}
\begin{verbatim}
int size()
\end{verbatim}
该方法可以返回数组的长度。如果数组为null,结果是未定义的。

\subsubsection{交错数组}
我们使用交错数组(Jagged Array)来达到多维数组的效果。交错数组就是数组的数组。
交错数组的申明方法和C\# 保持一致。
\begin{verbatim}
int[][] matrix;
\end{verbatim}
交错数组的创建语句如下:
\begin{verbatim}
int[][] graph = new int[3][];
graph[0] = null;
graph[1] = new int[10];
graph[2] = new int[30];
\end{verbatim}
需要先创建最外层数组的空间,然后再创建内层数组的空间。
\subsubsection{方便声明多维数组的文法糖}
\begin{verbatim}
int[][] matrix = new int[3][4];
\end{verbatim}
在14级里,这个文法糖产生的效果是未定义的。
支持交错数组的主流语言中,C\#不支持这么做,Java支持。
为了方便,从15级开始,我们就支持这种文法糖吧。

\section{类}
类的定义通过以下形式
\begin{verbatim}
class 类名 {
  类型1 字段名1;
  类型2 字段名2;
  类型3 函数名(参数序列) {
    各类语句  
  }
  类名() { // 可有可无的构造函数
    各类语句  
  }
}
\end{verbatim}

\subsection{用户自定义类的方法}
类似于函数。除了构造函数之外,都有返回值或void修饰符。

\subsection{构造函数}
构造函数是可选的。类定义中可以没有构造函数,
此时类的初始化行为是未定义的。

由于函数和类的重载是未定义的,
所以,出现在标准测试集里的程序,每个类最多应该只有一种构造函数,而且这个函数不带参数。
如果想实现构造函数重载,请尽管干吧。

\subsection{未定义的东西}
以下面向对象的基本特性都是未定义的
\begin{itemize}
\item private修饰符
\item 继承
\item 抽象类或接口
\item 多态
\item 成员的默认初始化表达式
\item 析构函数
\end{itemize}

\section{表达式}
\subsection{单目表达式}
单目表达式有常量,标识符变量名。等等
\subsection{双目表达式}
似乎不需要定义得太详细,大家都懂的。

\section{语句}
\subsection{声明语句}
\begin{verbatim}
类型 变量名;
\end{verbatim}
或者
\begin{verbatim}
类型 变量名 = 初始表达式;
\end{verbatim}
Java约定,如果没有初始表达式,则变量的初始值为0或null,我们对此类行为没有定义。
也就是说,标准测试集里的程序,应该保证变量在使用前先被赋值了,不然就不是一个合格的测试点。
\subsection{表达式语句}
\begin{verbatim}
表达式;
\end{verbatim}
\subsection{条件语句}
\begin{verbatim}
if (表达式1) 
  语句1 
else if (表达式2) 
  语句2
else 
  语句3
\end{verbatim}
判断表达式不能为空,且类型必须为bool
\subsection{循环语句}
\begin{verbatim}
while (表达式) 
  语句
\end{verbatim}
while的判断表达式不能为空,且类型必须为bool
\begin{verbatim}
for (表达式1;表达式2;表达式3) 
  语句
\end{verbatim}
for语句的表达式比较特殊,判断表达式可以为空,但如果不空的情况下,类型必须为bool
\subsection{跳转语句}
\begin{verbatim}
return 表达式;
break;
continue;
\end{verbatim}

\section{函数}

\subsection{函数定义}

\begin{verbatim}
类型 函数名 (参数序列) {
}
\end{verbatim}
Mx*没有方法声明函数的签名,也不支持在一个函数内嵌套申明另一个子函数或类。

\subsection{内建函数}
内建函数是指系统直接提供给用户的函数,不需要申明就可以使用。
\begin{verbatim}
void print(string str);
\end{verbatim}
向标准输出流中输出字符串str。
\begin{verbatim}
void println(string str);
\end{verbatim}
向标准输出流中输出字符串str,并在结尾处加上换行符。
\begin{verbatim}
string getString();
\end{verbatim}
从标准输入流里读取一行字符并返回。
\begin{verbatim}
int getInt();
\end{verbatim}
从标准输入流里读取一个整数并返回,如果输入流里并不是一个合法的整数,结果是未定义的。
\begin{verbatim}
string toString(int i);
\end{verbatim}
将一个整数转化为字符串。

\section{null}
null表示数组或者某个对象为空,不能用在int,bool上。
如果数组为null,再引用其某个下标,结果是未定义的。
string也不能赋值为null——虽然c\#和java是允许这么做的,但这样会不得不说明null和其他字符串的运算规定,这些叙述太复杂了。


\section{字符串}
\begin{verbatim}
// 错误!字符串不能赋值null
string str = null;
// 正确
string[] str_arr = null;
\end{verbatim}

\subsection{涉及字符串的运算符的语义}
\begin{itemize}
\item + 表示两个字符串的拼接
\item == 比较的是两个字符串内容是否完全一致(结构相等),而不是比较内存地址(引用相等)
\item < 比较字典序大小,其余关系运算符同理
\item 其他运算符的表现是未定义的
\end{itemize}

\subsection{字符串的内建方法}
\begin{verbatim}
int length()
\end{verbatim}
返回字符串长度。
\begin{verbatim}
string substring(int left, int right)
\end{verbatim}
返回下标从left开始到right结束的子串。
\begin{verbatim}
int parseInt();
\end{verbatim}
返回一个整数,这个整数应该是该字符串的最长前缀。
如果该字符串没有一个前缀是整数,结果未定义。如果该整数超界,结果也未定义。
\begin{verbatim}
int ord(int pos);
\end{verbatim}
返回字符串中的第pos位上的字符的ASCII码。下标从0开始编号。

\subsection{字符串常数的内建方法}
形如
\begin{verbatim}
"Four score and seven years ago".length();
\end{verbatim}
这样的表达式,所产生的效果是未定义的。
大家可以认为这是个语法错误,或者也可以返回该字符串常数的长度。



\section{左值的定义}

关于左值的定义,如果直接继承C语言对于左值的定义,会有一些问题
(因为C语言的结构类型是可以存在栈里的。所以直接复制会带来歧义)。

我们参考的是Java的官方说明手册。
\url{https://docs.oracle.com/javase/specs/jls/se7/html/jls-4.html#jls-4.12.3}
首先,Java的左值不称作左值,而是称为variable(见该文档15.1. Evaluation, Denotation, and Result的注释)。
简单来说Java的左值可以是以下几种:
\begin{itemize}
\item Method parameters name argument values passed to a method(函数的形参)
\item Local variables are declared by local variable declaration statements(局部变量)
\item An instance variable is a field declared within a class declaration(类的一个成员)
\item Array components are unnamed variables that are created and initialized to default values whenever a new object that is an array is created
(数组的一个元素)
\end{itemize}
我们要求语言至少能支持上述四种类型作为左值。
C++支持更多的左值形式,而我们的语言对更多类型的左值是未定义的。

\section{作用域规则 Scope Rule}
一个符号起作用的那一段程序区域称为这个变量的作用域。
\begin{itemize}
\item 在一段语句中,由\{和\}组成的块会引进一个新的作用域
\item 用户定义函数入口会引入一个新的作用域
\item 用户定义类的入口会引入一个新的作用域,该作用域里申明的所用成员,作用域为整个类。
\item 全局变量和局部变量不支持前向引用,作用域为声明开始的位置直到最近的一个块的结束位置
\item 函数和类的声明都应该在顶层,作用域为全局,支持前向引用(forward reference)。
\end{itemize}

\section{命名空间}
所有符号共享一个命名空间,所以在同一个作用域里,变量,函数,和class,都不能同名。
不同作用域的时候,内层作用域可以遮蔽外层作用域的名字。

函数重载或类的重载,都是未定义的。
\end{document}

